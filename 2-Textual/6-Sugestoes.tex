\chapter{Sugestões de trabalhos futuros}\label{Sugestões de trabalhos futuros}

\begin{itemize}

\item Levantamento de Custos: Propõe-se uma análise econômica completa do processo de forjamento, calculando todos os custos envolvidos - desde a matéria-prima e a fabricação das matrizes até as operações e tratamentos posteriores - para determinar o custo final por peça e confirmar a viabilidade financeira do projeto.
\item Simulação de Forjabilidade: Sugere-se o uso de simulação computacional para analisar o processo de forjamento, a fim de prever o fluxo do material, evitar defeitos, otimizar parâmetros e garantir que o componente final atinja as propriedades mecânicas e a integridade estrutural desejadas.
\item Análise Topológica: Recomenda-se a aplicação da otimização topológica para redesenhar o componente, buscando reduzir seu peso ao máximo sem comprometer a capacidade de carga (WLL). Este método utiliza algoritmo para encontrar a distribuição de material mais eficiente, gerando um design mais leve.

\end{itemize}